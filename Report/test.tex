%! Author = Miszka and Tamarka
%! Date = 10.03.2022

% Preamble
\documentclass[11pt]{article}

% Packages
\usepackage{float}
\usepackage[T1]{fontenc}
\usepackage{geometry}
\usepackage{parskip}
\usepackage{amsmath}
\usepackage{amsfonts}
\usepackage{amsthm}
\usepackage{amssymb}
\usepackage{titling}
%\usepackage{itemize}
\usepackage{enumerate}
\usepackage{multirow}
\usepackage{graphics}
\usepackage{caption}
\usepackage{array}
\setlength{\droptitle}{-2cm}
%\newgeometry{tmargin=1.9cm, bmargin=1.9cm, lmargin=1.7cm, rmargin=1.7cm}



\author{Tamara Frączek, Dominik Mika}
\title{Methods of classification and dimensionality reduction - Report 1}
\date{\today}

% Document
\begin{document}
\maketitle

\textbf{SVD}

We get a real $n \times d$ dimensional matrix $Z$.
(We may assume that $n \ge d$.)
The aim is to approximate $Z$ by a matrix of smaller rank.
Precisely, we want to find matrix $\tilde{Z}_r$ of rank $r$ and $r << rank(Z)$, so that $\|Z - \tilde{Z}_r\|$ is small.

From the theorem on the lecture we can present $Z$ as follows
\[Z = U \Lambda^{\frac{1}{2}} V^T,\]
where
\begin{itemize}
    \item $U$ is $n \times d$ matrix such that $U U^T = I$,
    \item $V$ is a $d \times d$ matrix such that $V V^T = I$,
    \item $\Lambda$ is a $d \times d$ diagonal matrix with values $\lambda_1, \dots, \lambda_d$ on diagonal such that $\lambda_1 \ge \dots \ge \lambda_d \ge 0$.
\end{itemize}


We construct $\tilde{Z}_r$ as
\[\tilde{Z}_r = U_r \Lambda_r^{\frac{1}{2}}V_r^T,\]
where
\begin{itemize}
    \item $U_r$ is $n \times r$ matrix such that it is a truncation of matrix $U$ to its first $r$ columns,
    \item $V_r$ is $r \times d$ matrix such that it is truncations of matrix $V$ to its first $r$ columns,
    \item $\Lambda_r$ is $r \times r$ diagonal matrix with $\lambda_1, \dots, \lambda_r$ on diagonal.
\end{itemize}
\end{document}