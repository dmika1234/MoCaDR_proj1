%! Author = Miszka and Tamarka
%! Date = 10.03.2022

% Preamble
\documentclass[11pt]{amsart}

% Packages
\usepackage{float}
\usepackage[T1]{fontenc}
\usepackage{geometry}
\usepackage{parskip}
\usepackage{amsmath}
\usepackage{amsfonts}
\usepackage{amsthm}
\usepackage{amssymb}
\usepackage{titling}
%\usepackage{itemize}
\usepackage{enumerate}
\usepackage{multirow}
\usepackage{graphics}
\usepackage{caption}
\usepackage{array}
\setlength{\droptitle}{-2cm}
%\newgeometry{tmargin=1.9cm, bmargin=1.9cm, lmargin=1.7cm, rmargin=1.7cm}

\DeclareMathOperator*{\argmin}{arg\,min}


\author{Tamara Frączek, Dominik Mika}
\title{Methods of classification and dimensionality reduction - Report 1}
\date{\today}

% Document
\begin{document}
\maketitle



\section{Statement of the problem}

We have the data containing information how users rate some movies.
Our task is to create a recommender system, so having only some data we want to predict all ratings.

\subsection{Description of the data}

The data contains information about around 100000 ratings - around 600 users rated around 9000 movies.
Our columns are: \textsf{userId}, \textsf{movieId} and \textsf{rating}.
We keep this data in two-dimensional matrix such that ...
If the user $i$ haven't rated the movie $j$ we leave $Z[i,j]$ empty.

\subsection{Quality of the system}


\section{Description of methods}

In this problem, we try different approaches.

First 3 methods, so SVD1, SVD2 and NMF get a $n \times d$ dimensional matrix $Z$ and approximate it by a different matrix $\tilde{Z}$.
Since we want somehow $\tilde{Z}$ to maintain only ''the most important'' information from $Z$, then the rank of $\tidle{Z}$ is to be much smaller than rank of $Z$.
Precisely, we want to find matrix $\tilde{Z}_r$ of rank $r$ ($r < rank(Z)$ and $r$ is a parameter), so that $\|Z - \tilde{Z}_r\|$ is small.

Because these methods are given a whole matrix, so they need the missing data to be imputed before performing.
The way we impute the data we describe later in the report in ??section??

\subsection*{SVD1}

Using SVD decomposition $Z = U \Lambda^{\frac{1}{2}} V^T$ we construct $\tilde{Z}$ as
\[\tilde{Z}_r = U_r \Lambda_r^{\frac{1}{2}}V_r^T,\]
where $\Lambda_r$ contains $r$ biggest eigenvalues of $Z$ and $U_r$, $V_r$ contains only columns corresponding to those eigenvalues.

\subsection*{SVD2}

SVD2 is an iterative method using SVD1, so we perform SVD1 on matrix $Z$, then on the result of first SVD1 and so on.
Since $\tilde{Z}$ can have different values than actual in elements that we actually now (so these from training set), so we in every step we have to make a correction and  impute the real values there.
We stop when...

\subsection*{NMF}
This time we approximate $Z$ with $\tilde{Z}_r = W_r H_r $, where $W_r$ and $H_r$ are matrices with non-negative elements ($W_r$ has $r$ columns and $H_r$ has $r$ rows).
Precisely, we look for such $W_r$ and $H_r$ that $\|Z - W_r H_r \|^2$ is the smallest, where $\|A\|^2 = \sum_{i, j} A_{ij}^2$.

\subsection*{SGD}


\section{Implementation}
\subsection*{Performing methods}
\subsection*{Choosing parameters}

\section{Results}



\end{document}